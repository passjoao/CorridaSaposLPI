Projeto da disciplina Linguagem de programação I ~\newline
Aplicação dos conceitos de Programação Orientada a Objetos (P\+OO)

\subsection*{Autor}

João Victor Soares Oliveira

\subsubsection*{\href{https://github.com/passjoao/CorridaSaposLPI}{\tt Link para acesso no gitgub}}

\#\# Compilar 
\begin{DoxyCode}
$make
\end{DoxyCode}
 \#\# Executar 
\begin{DoxyCode}
$.bin/execute
\end{DoxyCode}
 \subsection*{Descrição}

O projeto busca implementar uma corrida de sapos criados pela classe \hyperlink{classSapo}{Sapo}, no qual, o usuário poderá fazer as seguintes funcionalidades \tabulinesep=1mm
\begin{longtabu} spread 0pt [c]{*{2}{|X[-1]}|}
\hline
1. Mostrar estatísticas dos sapos&O usuário podera ver todos os sapos salvos no vector sapos. \\\cline{1-2}
2. Mostrar estatísticas das pistas&O usuário podera ver todos as pistas salvas no vector pistas. \\\cline{1-2}
3. Criar um sapo&O usuário poderá criar um sapo adcionando-\/o ao vector sapos. \\\cline{1-2}
4. Criar uma pista&O usuário poderá criar uma pista adcionando-\/a ao vector pistas. \\\cline{1-2}
5. iniciar uma corrida&O usuário poderá iniciar a corrida. \\\cline{1-2}
\end{longtabu}


{\bfseries 1.} O usuário poderá ver os dados de todos os sapos do vector sapos, exibindo o Id, Nome, Distância percorrida, quantidade de pulos dados, quantidade de provas disputadas, vitórias, empates e o salto máximo de sapo.

{\bfseries 2.} O usuário poderá ver os dados de todas as pistas do vector pistas, exibindo o Id da pista e a distância dela.

{\bfseries 3.} O usuário poderá criar um sapo pela função criar\+Sapo(), o usuário definirá o salto máximo e o nome do sapo, o programa colocará o Id do sapo como o último adicionado + 1, e os valores restantes como 0 e adiciona ao vector sapos.

{\bfseries 4.} O usuário poderá criar uma pista pela função criarpista(), definindo o tamanho da mesma e o progama adiciona ao vector pista

{\bfseries 5.}Ao iniciar a corrida, será mostrado os sapos que irão para a corrida, será exibido cada loop do programa o sapo, o ID e o tamanho do pulo dado, no qual os sapos pulam um de cada vez por ciclo até que todos cheguem a linha de chegada, definida pelo tamanho da pista


\begin{DoxyItemize}
\item 
\end{DoxyItemize}

Antes do usuário entrar no menu de execuções, o programa irá ler dois arquivos .txt, contendo um banco de dados dos sapos e pistas salvando em seus respectivos vectors


\begin{DoxyItemize}
\item 
\end{DoxyItemize}

Ao sair do programa, todos os sapos e pistas criados serão salvos nos arquivos .txt